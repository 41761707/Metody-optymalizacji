\documentclass[12pt, a4paper]{report}
\usepackage{polski}
\usepackage[utf8]{inputenc}
\usepackage{hyperref}
\hypersetup{
    colorlinks=true,
    linkcolor=black,
    urlcolor=blue,
    citecolor=black
}
\usepackage[compact]{titlesec}
\titleformat{\chapter}{}{}{0em}{\bf\LARGE}
\usepackage{lipsum}


\renewcommand{\contentsname}{Spis treści}
\renewcommand{\refname}{Bibliografia}

\title{ Job shop scheduling}
\author{Opracowanie wybranego problemu optymalizacyjnego \\
Radosław Wojtczak, numer indeksu: 254607}
\date{28.05.2023}
\begin{document}
\maketitle
\tableofcontents

\chapter{Wprowadzenie}
  \section{Opis problemu}
  
  \textbf{Job shop scheduling problem (JSSP)}  jest jednym z popularniejszych problemów optymalizacyjnych. 
  Jego celem jest znalezienie optymalnego sposobu alokacji współdzielonych zasobów w czasie względem 
  konkurujących ze sobą aktywności, tak aby czas realizacji aktywności był jak najmniejszy.
  W ogólności wskazany problem definiuje się w następujący sposób:
  \begin{itemize}
    \item liczba $n$ określą liczbę prac (ang. jobs), które muszą zostać zrealizowane
    $(J_{1}, \dots, J_{n})$
    \item liczba $m$ okreslą liczbę współdzielonych zasobów (często w odniesieniu 
    do zasobów używa się określenia \textit{maszyny}) $(M_{1}, \dots, M_{m})$
    \item celem jest utworzenie takiego harmonogramu, aby czas realizacji zadań
    był jak najmniejszy (czas ten najczęściej w literaturze określany jest mianem \textbf{makespan})
  \end{itemize}
  Termin \textit{job shop} pochodzi od charakterstycznego modelu produkcji, w którym różne zadania (jobs) przechodzą 
  przez różne etapy produkcji w zakładzie produkcyjnym (shop). Na podstawie pochodzenia nazwy problemu łatwo zauważyć,
  iż jestem to problem, który wywodzi się bezpośrednio z potrzeb życia codziennego, stąd jest on jednym z pierwszych 
  problemów kombinatorycznych, dla którego przeprowadzono analizę konkurencyjności (ang. competetive analysis) \cite{Graham_competetive} \\
  Ze względu na szeroką gamę zastosowań, na przestrzeni lat wytworzyło się wiele odmian wskazanego problemu.
  W podstawowych rozważaniach każde z zadań składa się ze zbioru $k$ operacji $(O_{1}, \dots, O_{k})$ które muszą zostać wykonane 
  w odpowiedniej kolejności (co znane jest jako \textit{precedence constraint}). Każda z operacji 
  wykonuje się określony czas oraz każda z maszyn może wykonywać jedną operację w danym momencie. Istnieje
  wiele odmian problemu, takie jak:
  \begin{itemize}
    \item \textit{flexible job shop with duplicate machines} pozwalające na istnienie takich samych maszyn, czego naturalnym
    rozszerzeniem jest \textit{flexible job shop, FJSSP} który dopuszcza istnienie wielu (lub nawet wszystkich) identycznych maszyn, 
    bądź identycznych grup maszyn
    \item odmiany, w których nakładane są ograniczenia na maszyny (wymuszenie przestojów, przerw w działaniu)
    \item odmiany, w których nakładane są ograniczenia na prace ($i-ta$ praca może rozpocząć się dopiero,
    gdy $j-ta$ praca się zakończy)
  \end{itemize}

\chapter{Przykładowe reprezentacje problemu}
  \section{Graf rozłączny}
  Najpopularniejszy sposób reprezentacji problemu job shop scheduling wykorzystuje strukturę 
  o nazwie \textbf{graf rozłączny} (ang.disjunctive graph) \cite{disjunctive_graph}.
  Graf rozłączny jest grafem skierowanym $ G =(V,C \: \cup \: D)$, gdzie:
  \begin{itemize}
    \item V oznacza zbiór 
    wierzchołków odpowiadający zadaniom (bądź operacjom, które wchodzą w skład zadań, w zależności od rozpatrywanej instancji problemu),
    który zawiera w sobie dodatkową, sztucznie wytworzoną parę wierzchołków: początkowy (\textit{start}) oraz końcowy (\textit{sink}). 
    Oba wyróżnione wierzchołki mają zerowy czas wykonywania, istnieją jedynie po to, aby ustalić relację wykonywania zadań (zadanie startowe
    musi zostać wykonany przed wszystkimi innymi zadaniami, natomiast zadanie końcowe zostaje wykonane wtedy, gdy zostaną ukończone wszystkie pozostałe czynności).
    \item C to zbiór łuków łączących (conjunctive arcs), 
    które odzwierciedlają początkowe ograniczenia kolejności, 
    łączące każde dwie kolejne operacje tego samego zadania
    \item Nieskierowane krawędzie rozłączne należące do zbioru D 
    łączą wzajemnie nieuporządkowane zadania, 
    które wymagają tej samej maszyny do wykonania 
    (krawędź rozłączna może być reprezentowana przez dwa przeciwstawne łuki skierowane). 
    Każdy łuk jest oznaczony dodatnią wagą równą czasowi wykonania zadania, 
    od którego łuk się zaczyna.
  \end{itemize}
  W modelu grafu rozłącznego (disjunctive graph), znalezienie optymalnego 
  porządku wykonaia wszystkich zadań na maszynach
  równoważne jest z wyborem jednego łuku w każdej dysjunkcji, czyli zamienieniem 
  każdej nieskierowanej krawędzi dysjunkcyjnej na skierowaną krawędź łączącą. 
  Poprzez ustalenie kierunków wszystkich krawędzi dysjunkcyjnych, 
  określa się kolejność wykonywania wszystkich konfliktujących zadań wymagających 
  tej samej maszyny, co prowadzi do uzyskania kompletnego harmonogramu.
  Ponadto, uzyskany graf musi być acykliczny, a jeśli jest optymalny, 
  to długość najdłuższej ścieżki od źródła do ujścia jest minimalna. 
  Ta najdłuższa ścieżka określa \textit{makespan}

  \section{Zdarzeniowy graf sieciowy}
  Graf rozłączny zawiera jedynie statyczne informacje o problemie, takie jak czasy trwania
  operacji, czy ograniczenia odnośnie kolejności, bez uwzględnienia dynamicznie
  zmieniających się informacji w ramach konstruowania harmonogramu. 
  Zdarzeniowy graf sieciowy (Event-based Network Graph) wprowadza do wezłów grafu dodatkowe 
  informacje, tak aby graf G reprezentował bieżący stan JSSP podczas planowania.
  \cite{fjssp} \cite{graf_sieciowy}

  \section{Model matematyczny}
  W celu reprezentacji problemu z rodziny job shop popularne jest wykorzystanie modelu 
  matematycznego, który może być pomocny chociażby podczas próby rozwiązania wspomnianego 
  zagadnienia przy pomocy technik programowania matematycznego, w którego skład wchodzi programowanie liniowe,
  czy programowanie całkowitoliczbowe.
  Zdefiowawszy następującą instancję problemu: \\
  $jobs = \{1, \dots, n\} $ - zbiór składający się z n zadań, $m$- liczba identycznych maszyn, $(i,j)$-
  relacja pierwszeństwa między zadanami, która oznacza, iż zadanie $j$ może rozpocząć się dopiero wtedy, gdy 
  zadanie $j$ zostanie ukończone.  Celem zadania jest znalezienie dopuszczalnego harmonogramu, który
  minimalizuje całkowity czas potrzebny do wykonania wszystkich zadań. Czas
  ten oznaczono jako $c\_max$.\\
  W poniższy sposób prezentuje się jej model programowania liniowego:
  \begin{itemize}
      \item Wymagany czas nie może być mniejszy niż suma rozpoczęcia ostatniej pracy na maszynie oraz czasu jej trwania
      \begin{equation}
          (\forall {j \in jobs}) (\sum_{t \in times} x_{jt} * (p_{j}+t)) \leq c\_max
      \end{equation}
      \item Każda praca może mieć tylko jeden moment rozpoczęcia 
      \begin{equation}
          (\forall {j \in jobs}) (\sum_{t \in times} x_{jt} = 1 )
      \end{equation}
      \item Zachowanie relacji pierwszeństwa między zadaniami
      \begin{equation}
          (\forall {(j,k) \in precedence)}(\sum_{t \in times} (p_{j} + t)*x_{jt} \leq \sum_{t \in times} t*x_{kt}) 
      \end{equation}
      \item W jednej jednostce czasu nie może być więcej aktywny zadań niż dostepnych maszyn
      \begin{equation}
          (\forall {t \in times}) (\sum_{j \in jobs} \sum_{s = max\{0,t-p_{j}+1 \}}^ {t} x_{js} ) \leq m
      \end{equation}
  \end{itemize}
  gdzie zbiór \textit{times} oznacza horyzont zdarzeń, czyli dyskretny podział czasu na sloty, w których dana maszyna może obsłużyć zadanie, a 
  \begin{equation}
    x_{jt} : j \in jobs, t \in times, x_{jt} \in \{0,1\}
  \end{equation}
  jest zmienną decyzyjną w formie macierzy przechowującej momenty rozpoczęcia każdego z zadań.
  Powyższy opis wraz z naturalną definicją funkcji celu jako minimalizacji zmiennej 
  \begin{equation}
    min \: c\_max
  \end{equation}
  tworzy model programowania liniowego dla pewnej odmiany problemu JSSP.

\chapter{Zastosowania}
  \section{Zarządzanie zadaniami na wielu procesorach}
    Instrukcje programu komputerowego można traktować jako zadania, a maszyny jako procesory w komputerze.
    Zgodnie z powyższym możemy założyć, że istnieją pewne zadania, które muszą zostać wykonanie jednocześnie na dwóch bądź więcej procesorach, co
    sprowadza się do ustalenia harmonogramu zadań tak, aby każdy z procesorów potrzebnych do realizacji danego zdania w danym momencie był dostepny oraz 
    aby żaden procesor nie realizował dwóch zadań jednocześnie. Jest to perfekcyjny przykład, w którym można wykorzystać metody przedstawione w JSSP, co pokazują
    prace \cite{jobshop_usage_multiprocessor}, \cite{jobshop_usage_multiprocessor_2} 
    
  \section{Harmonogramowanie w kolejnictwie}
    Problem harmonogramowania w kolejnictwie można modelować jako problem job-shop poprzez podzielenie sieci
    kolejowej na segmenty torów. Jako zadanie traktuje się pociąg, który musi dostać się z pewnego 
    miejsca początkowego do celu. Jako operacje uznaje się segmenty torów na trasie, co bezpośrednio wynika z chrakterystyki kolei- 
    pociągi nie mogą wykonywać manewrów wymijania, więc istotnym jest, aby tylko jeden pociąg w danej jednostce czasu 
    okupował daną część trasy. Czas przetwarzania operacji to czas potrzebny na przejazd przez segment toru. Cała problematyka sprowadza się do utworzenia harmonogramu 
    w taki sposób, aby zmimializować czas, w którym pociągi oczekują na możliwość wjechania do odpowiedniego segmentu torów. \cite{kolejnictwo_jobshop}

  \section{Zarządzanie produkcją w fabryce}
  Ze względu na wielokrotne wykorzystanie analogii między maszynami a zadaniami, zarządzanie fabryką jest głównym obszarem zastosowania problemu optymalizacyjnego job shop. 
  W kontekście zarządzania fabryką,  problem job shop scheduling dotyczy efektywnego harmonogramowania zadań na maszynach w celu minimalizacji 
  takich rzeczy jak czasu realizacji \cite{czas_jobshop}, zużycia energii \cite{energia_jobshop} czy produkcji odpadów bądź zanieczyszczeń \cite{zanieczyszczenia_jobshop}.


\chapter{Rozwój metodologii rozwiązań}
  \section{Programowanie matematyczne}
    Pierwsze prace nad problemem JSSP skoncentrowały się na rozwoju matematycznych modeli opisujących problem i 
    opracowaniu algorytmów optymalizacyjnych. Jednym z wczesnych podejść było użycie metod programowania całkowitoliczbowego 
    (integer programming) \cite{IP} \cite{IP_2}, które opierały się na sformułowaniu problemu jako zadania minimalizacji 
    pewnej funkcji celu przy pewnych ograniczeniach. Ponadto wykorzystano takie techniki rozwiązywania problemów jak 
    programowanie liniowe (linear programming), programowanie liniowe całkowitoliczbowe
    (mixed-integer linear programming) czy programowanie ograniczeń (constraint programming) \cite{CP_1} \cite{CP_2}. 
    Te metody umożliwiały bardziej precyzyjne modelowanie i rozwiązywanie problemu, uwzględniając dodatkowe ograniczenia i zależności 
    między operacjami.

    \section{Heurystyki}
    Wraz z rozwojem technologii i rosnącą dostępnością komputerów, pojawiły się nowe metody rozwiązywania problemu JSSP. 
    Algorytmy heurystyczne, takie jak przeszukiwanie lokalne (local search) i algorytmy genetyczne (genetic algorithms), \cite{GA_1} \cite{GA_2}
    zaczęły być wykorzystywane do rozwiązywania tego problemu. Metody te mogły generować satysfakcjonujące rozwiązania w krótszym czasie,
    choć niekoniecznie optymalne.

    \section{Metaheurystyki, algorytmy ewolucyjne}
    Wraz z postępem naukowym, zastosowanie innych metod, takich jak metaheurystyki, systemy wieloagentowe i algorytmy ewolucyjne, 
    stało się popularne w rozwiązywaniu problemu JSSP \cite{MH_1} \cite{MH_2}. Te zaawansowane techniki algorytmiczne i sztucznej inteligencji były w stanie 
    generować coraz lepsze i bardziej optymalne harmonogramy, uwzględniając różne cele i ograniczenia.

    \section{Sztuczna inteligencja}
    Obecnie, wraz z rozwojem technologii i wzrostem ilości dostępnych danych, popularne stało się wykorzystywanie uczenia maszynowego i 
    sztucznej inteligencji do rozwiązywania problemu JSSP. Metody oparte na głębokim uczeniu (deep learning) są 
    stosowane do generowania optymalnych harmonogramów, a także do automatycznego uczenia się z wcześniejszych rozwiązań. \cite{AI_1} \cite{AI_2} \cite{AI_3}

\chapter{Podsumowanie}
  Job shop scheduling jest istotnym problemem kombinatorycznym, 
  który znajduje swoje zastosowanie w wielu dziedzinach. Z tego względu 
  wielu specjalistów poświęciło znaczną część swojego zawodowego życia aby 
  rozwijać metody i technologie wykorzystywane przy 
  tworzeniu nowych sposobów rozwiązań opisywanego problemu. 
  Począwszy od opracowania 
  matematycznych modeli, takich jak programowanie całkowitoliczbowe i 
  programowanie liniowe, aż po rozwinięcie algorytmów optymalizacyjnych, 
  heurystyk, czy wykorzystania uczenia maszynowego, 
  możliwe stało się szybkie i skuteczne poszukiwanie optymalnych 
  lub bliskich optymalnych rozwiązań. Osiągnięcia w dziedzinie job shop scheduling 
  mają ogromne znaczenie dla różnych sektorów i branż, takich jak produkcja, 
  logistyka, czy transport. Poprawa planowania i harmonogramowania zadań 
  przekłada się na zwiększenie wydajności, redukcję czasu przestoju maszyn, 
  minimalizację kosztów i poprawę jakości produkcji oraz środowiska, 
  co bezpośrednio wpływa na lepsze i łatwiejsze funkcjonowanie 
  ludzi w otaczającym ich świecie.


  \clearpage
  \phantomsection
  \addcontentsline{toc}{section}{Bibliografia}
  \bibliographystyle{plain}
  \bibliography{literatura}
\end{document}