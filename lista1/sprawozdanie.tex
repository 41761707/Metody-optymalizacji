\documentclass[a4paper,14pt]{report}
\usepackage[utf8]{inputenc}
\usepackage[T1]{fontenc}
\usepackage{titlesec}
\usepackage{float}

\renewcommand{\contentsname}{Spis treści}

\titleformat{\chapter}[display]
  {\normalfont\bfseries}{}{0pt}{\Huge}

\title{Metody optymalizacji Lista 1}
\author{Radosław Wojtczak, numer indeksu: 254607}
\date{02.04.2023}
\begin{document}
\maketitle
\tableofcontents
\chapter{Zadanie 1}
\section{Wprowadzenie}
    Pierwszym zadaniem było przeprowadzenie testów na dokładność oraz odporność
    algorytmów LP z użyciem specjalnie spreparowanego przykładu wykorzystującego
    macierz Hilberta.
    Ze względu na swoją budowę, macierz ta
    powoduje złe uwarunkowanie zadania, nawet dla niewielkich wartości parametru
    $n$, który oznacza rozmiar problemu.
\section{Model}
    \subsection{Zmienne decyzyjne}
        \begin{equation}
            x \geq 0
        \end{equation}
        Co oznacza, iż każdy z elementów wektora nie może być liczbą ujemną.
    \subsection{Ograniczenia}
        \begin{equation}
            A \boldmath x = b
        \end{equation}
        Dla elementów macierzy zdefiniowanych w następujący sposób.
        \begin{equation}
            a_{i,j} = \frac{1}{i+j-1}, i,j=1,..,n \\
        \end{equation}
        \begin{equation}
            c_{i} = b_{i} =\sum_{j=1}^{n} \frac{1}{i+j-1}, i,j=1,..,n
        \end{equation}
    \subsection{Funkcja celu}
        \begin{equation}
            min \: \boldmath c^{T}x
        \end{equation}
\section{Wyniki}
    \begin{table}[H]
        \centering
        \begin{tabular}{|c | c |} 
        \hline
        n & $\frac{|| x - \hat x||_{2}}{|| x ||_{2}}$\\
        \hline
        3 & 0.000000\\
        4 & 0.000000\\
        5 & 0.000000\\
        6 & 0.000000\\
        7 & 0.000000\\
        8 & 0.514059\\
        9 & 0.682911\\
        10 & 0.990388\\
        11 & 0.952197\\
        12 & 1.292507\\
        \hline
        \end{tabular}
        \caption{Wartości błędów względnych dla wskazanych wartości parametru n}
    \end{table}
    Zauważamy, iż ostatni rozmiar problemu, dla którego można rozwiązać zadanie 
    z dokładnością do co najmniej 2 cyfr jest równy \textbf{7}. Od tej wartości błąd względny jest niezerowy,
    ponadto dla rosnący wartości parametru n, zwiększa się również błąd, co wskazuje na to, iż 
    im większy rozmiar problemu tym większa niedokładność w otrzymanych wynikach.
    Wiedząc, iż rozwiązaniem rozpatrywanego równania jest wektor \textbf{x}, gdzie $x_{i} = 1, i=1,...,n$ poniżej 
    zaprezentowano wyniki otrzymane dla dwóch wybranych rozmiarów problemów:
    \begin{table}[H]
        \centering
        \begin{tabular}{|c | c | c |} 
        \hline
        Indeks & n=7 & n=8\\
        \hline
        x[0] & 1 & 1.00006\\
        x[1] & 1 & 0.996503\\
        x[2] & 1 & 1.04662 \\
        x[3] & 1 & 0.74359 \\
        x[4] & 1 & 1.6993\\
        x[5] & 1 &  0\\
        x[6] & 1 & 1.71795\\
        x[7] & --- &  0.795918\\
        \hline
        \end{tabular}
        \caption{Współrzędne wektorów dla wybranych rozmiarów problemu}
    \end{table}
\section{Wnioski}
    Realizując problemy przy użyciu narzędzi programowania liniowego należy zwrócić uwagę na uwarunkowanie zadania, gdyż
    ma ono znaczny wpływ na otrzymywane wyniki. Im zadanie jest gorzej uwarunkowane tym otrzymane wyniki odbiegają bardziej 
    od wartości rzeczywistych.

\chapter{Zadanie 2}
\section{Wprowadzenie}
    Pewna firma transportowa miała problem z poprawnym rozdysponowaniem 
    dźwigów samojezdnych w swoich 7 siedzibach umiejscowionych w 
    południowo-zachodniej części Polski. W ramach zadania dana była 
    tabela prezentująca nadmiar jak i niedobór dźwigów samojezdnych,
    występujących w dwóch typach, w każdej z placówek firmy
    Przy pomocy narzędzia Google Maps ustalono odległości między placówkami firmy,
    które zostały zaprezentowane w poniższej tabeli:
    \begin{table}[H]
        \centering
        \begin{tabular}{|c | c | c | c | c | c | c | c |} 
        \hline
         & Opole & Brzeg & Nysa & Prudnik & Strzelce & Koźle & Racibórz \\
        \hline
        Opole & 0 & 43 & 58 & 65 & 34 & 53 & 78 \\
        Brzeg & 43 & 0 & 53 & 80 & 78 & 99 & 126 \\
        Nysa & 58 & 53 & 0 & 33 & 86 & 75 & 89 \\
        Prudnik & 65 & 80 & 33 & 0 & 69 & 48 & 62 \\
        Strzelce & 34 & 78 & 86 & 69 & 0 & 26 & 61 \\
        Koźle & 53 & 99 & 75 & 48 & 26 & 0 & 36 \\
        Racibórz & 78 & 126 & 89 & 62 & 61 & 36 & 0 \\
        \hline
        \end{tabular}
        \caption{Odległości między siedzibami firmy}
    \end{table}
    Ze względu estetycznych nazwa miejscowości \textit{Strzelce Opolskie} w tabeli, jak i w dalszej 
    części sprawozdania została zredukowana do \textit{Strzelce}.
    Ponadto koszt przeniesienia dźwigów z miasta A do miasta B jest zależne od 
    jego typu. Koszt transportu dźwigu typu II jest o $20\%$ wyższy, niż koszt 
    transportu dźwigu I. Ponadto dźwig typu I może zostać zastąpiony dźwigiem typu II,
    jednakże konwersja jest jedynie jednostronna.
    Celem zadania jest wyznaczenie planu, zgodnie z którym firma powinna dokonać 
    przemieszczania dźwigów między swoimi siedzibami, aby potrzeby każdej z placówek 
    były zaspokojone, przy minimalnym koszcie transportu.

\section{Model}
    W skład modelu wchodzą następujące elementy:
    \begin{itemize}
        \item Zbiór miast, określany mianem \textbf{Cities}, który zawiera
        wszystkie miasta, w których rozpatrywana firma posiada swoją placówkę
        \item Zbiór typów dźwigów, określany mianem \textbf{Cranes}
    \end{itemize}
    Ponadto zdefiniowano następujące parametry:
    \begin{itemize}
        \item Parametr $excess(crane \in Cranes, city \in Cities)$, która definiuje nadmiar dźwigów dla wskazanego miasta
        \item Parametr $deficiency(crane \in Cranes, city \in Cities)$, która definiuje niedobór dźwigów dla wskazanego miasta
        \item Parametr $transport\_cost(crane \in Cranes)$ definiujący koszt transportu dźwigów o wskazanym typie
        \item Parametr $could-replace(crane \in Cranes)$ wskazujący, który z dźwigów może zostać zastąpiony przez dźwig innego typu
    \end{itemize}

    \subsection{Zmienne decyzyjne}
        Zmienną decyzyjną we wskazanym modelu jest liczba dźwigów, którą należy przetransportować z miasta \textbf{c1} do miasta \textbf{c2}
        \begin{equation}
            x_{c1,c2,crane} c1,c2 \in Cities, crane \in Cranes
        \end{equation}
    \subsection{Ograniczenia}
        W ramach zadania podane zostały następujące ograniczenia:
        \begin{itemize}
            \item Liczba dźwigów, która została przetransportowana do wskazanego miasta 
            powinna być równa niedoborowi, który wskazane miasto posiadało.
            \begin{equation}
                ( \forall c1 \in Cities) (\sum_{c2 \in Cities, c \in Cranes} x_{c1,c2,c} = deficiency(c,c1))
            \end{equation}
            \item Dźwig typu II nie może zostać zastąpiony przez dźwig typu I
            \begin{equation}
                ( \forall c1 \in Cities) ( x_{c1,c2,c_{II}} \geq deficiency(c_{II},c1))
            \end{equation}
            \item Liczba nadmiarowych dźwigów musi zostać zniwelowana dla każdego z miast
            \begin{equation}
                ( \forall c1 \in Cities, c \in Cranes) (\sum_{c2 \in Cities} x_{c1,c2,c} = excess(c,c1))
            \end{equation}
        \end{itemize}
    \subsection{Funkcja celu}
        Celem tego zadania było zminimalizowanie funkcji celu:
        \begin{equation}
         min :\ \sum_{c1,c2 \in Cities, c \in Cranes} distance(c1,c2) * x_{c1,c2,crane}  * transport\_cost(crane) 
        \end{equation}
        Wynika to z faktu, iż firma chce zapłacić jak najmniej za transport dźwigów, który zgodnie z powyższym opisem zadanej sytuacji jest zależny od dystansu, 
        który musi zostać pokonany.
\section{Wyniki}
    Uruchomienie programu, który implementuje wskazaną sytuację, skutkuje otrzymaniem następującego planu transportu dźwigów:
    \begin{table}[H]
        \centering
        \begin{tabular}{|c | c | c | c |} 
        \hline
        Z & Do & Liczba & Typ\\
        \hline
        Opole & Brzeg & 4 & I \\
        Opole & Koźle & 3 & I \\
        Brzeg & Brzeg & 1 & II \\
        Nysa & Opole & 2 & II \\
        Nysa & Brzeg & 5 & I \\
        Nysa & Prudnik & 1 & I \\
        Prudnik & Prudnik & 3 & II \\
        Prudnik & Strzelce & 4 & II \\
        Prudnik & Koźle & 2 & II \\
        Prudnik & Racibórz & 1 & II \\
        Strzelce & Koźle & 5 & I \\
        \hline
        \end{tabular}
        \caption{Wykreowany plan transportu dźwigów między placówkami firmy}
    \end{table}
    Sytuacja występująca chociażbym w trzecim wierszu planu wskazuje na sytuację, w której doszło do zamiany typów dźwigu (gdyż zgodnie z założeniami, dźwig typu I mógł 
    zostać zastąpiony przez dźwig typu II)
    Wartość funkcji celu wyniosła: \textbf{1419}
    Ponadto, założenie całkowitoliczbowości zmiennych decyzyjnych okazało się być \textbf{niepotrzebne} w tym zadaniu.
    Pomijając wskazane ograniczenie program zwrócił ten sam plan z tą samą wartością funkcji celu.
\section{Wnioski}
    Programowanie liniowe jest użytecznym narzędziem rozwiązującym w skuteczny sposób problemy natury rzeczywistej.
\chapter{Zadanie 3}
\section{Wprowadzenie}
    W ramach zadania rozpatrzono pewną rafinerię, która dysponuje jednostką destylującą, która pozwala 
    otrzymać cztery rodzaje produktów: paliwa silnikowe, oleje, destylaty ciężkie oraz resztki. Ponadto 
    posiada jednostkę reformowania oraz jednostkę krakowania katalitycznego, która może przetwarzać destylaty ciężkie.
    W ramach zadania należało zminimalizować koszt produkcji wskazanej rafinierii, która posiada do dyspozycji 
    dwa typy ropy, ropę \textbf{B1} oraz robię \textbf{B2}. Wydajności procesów, oraz koszty operacji były dane 
    w ramach treści zadania.
\section{Model}
    W skład modelu wchodzą następujące elementy:
    \begin{itemize}
        \item Zbiór typów ropy o nazwie \textbf{Oils}, w skład którego wchodzą dwa wyżej wymienione typy: $\{B1,B2\}$
        \item Zbiór elementów otrzymanych w ramach destylacji o nazwie \textbf{Types}, w skład którego wchodzą: $\{Petrol,Oil,Distill,Leftovers\}$
    \end{itemize}
    \subsection{Zmienne decyzyjne}
        W ramach modeulu rozpatrujemy następujące zmienne decyzyjne
        \begin{itemize}
            \item b1 - Liczba ton ropy B1, która została przetworzona przez rafinerię
            \item b1\_ho - Liczba ton oleju z ropy B1, która została przeznaczona na domowe paliwa
            \item b1\_he - Liczba ton oleju z ropy B1, która została przeznaczona na ciężkie paliwa
            \item b1\_d\_c - Liczba ton destylatu z ropy B1, która została przeznaczona na krakowanie
            \item b1\_d\_h - Liczba ton destylatu z ropy B1, która został preznaczony na ciężkie paliwa
            \item b2 - Liczba ton ropy B2, która została przetworzona przez rafinerię
            \item b2\_ho - Liczba ton oleju z ropy B2, która została przeznaczona na domowe paliwa
            \item b2\_he -  Liczba ton oleju z ropy B2, która została przeznaczona na ciężkie paliwa
            \item b2\_d\_c - Liczba ton destylatu z ropy B2, która została przeznaczona na krakowanie
            \item b2\_d\_h - Liczba ton destylatu z ropy B2, która został preznaczony na ciężkie paliwa
        \end{itemize}
    \subsection{Ograniczenia}
        Treść zadania determinowała następujące ograniczenia:
        \begin{itemize}
            \item Ograniczenie dotyczące liczby wyprodukowanego paliwa silnikowego
            \begin{equation}
                b1_{petrol} * b1 + b2_{petrol} * b2 + 0.5 * d\_c + 0.5 * d\_h \geq  200000 
            \end{equation}
            \item Ograniczenie dotyczące liczby wyprodukowanego domowego paliwa
            \begin{equation}
                b1\_ho + b2\_ho + 0.2 * b1\_d\_c + 0.2 * b2\_d\_c \geq 400000
            \end{equation}
            \item Ograniczenie dotyczące liczby wyprodukowanego ciężkiego paliwa
            \begin{equation}
                b1\_he + b1\_d\_h + b1_{d} * b1 + b2\_he + b2\_d\_h + b2_{d} * b2 + 0.06 * b1\_d\_c + 0.06 * b2\_d\_c \geq 250000
            \end{equation}
            \item Ograniczenia dotyczące zawartości siarki \\
            $0.002 * b1\_ho + 0.003 * b1\_d\_c * 0.2 + 0.012 * b2\_ho + 0.025 * b2\_d\_c * 0.2 \leq (b1\_ho + b2\_ho + b1\_d\_c * 0.2 + b2\_d\_c * 0.2) * 0.005$
            \item Ograniczenie odnośnie ilości oleju
            \begin{equation}
                b1\_ho + b1\_he = b1_{oil} * b1
            \end{equation}
            \begin{equation}
                b2\_ho + b2\_he = b2_{oil} * b2
            \end{equation}
            \item Ograniczenie odnośnie ilości destylatu
            \begin{equation}
                b1\_d\_c + b1\_d\_h = b1_{d} * b1
            \end{equation}
            \begin{equation}
                b2\_d\_c + b2\_d\_h = b2_{d} * b2
            \end{equation}
        \end{itemize}
        Ze względów estetycznych nie wykorzystano zmiennych aby zaprezentować przeliczniki ustalone w ramach zadnia, 
        zostały one wprost zaaplikowane do równań.
    \subsection{Funkcja celu}
        Funkcja celu prezentuje się następująco:
        \begin{equation}
            min :\ (1300+10) * b1 + (1500+10) * b2 + 20 * b1\_d\_c + 20 * b2\_d\_c
        \end{equation}
        Co sprowadza się do minimalizacji kosztów przetwarzania ropy
\section{Wyniki}
        Program po uruchomieniu wygenerował wartość funkcji kosztu równą: \textbf{1345943600.86768}.
        W ramach operacji wykonywanych w rafinerii wykorzystano \textbf{1026030.36876356} ton ropy B1 i 
        0 ton ropy B2. Ostatecznie otrzymano \textbf{381562} ton oleju wykorzystanego na domowe paliwa, 
        \textbf{28850.3} ton oleju wykorzystanego na ciężkie paliwa. Ponadto przeznaczono \textbf{61713.7} ton 
        destlatu na ciężkie paliwa oraz \textbf{92190.9} ton na krakowanie katalityczne.
\section{Wnioski}
        Powyższy test jest kolejny praktyczny przykładem zastosowania programowania liniowego 
        w rozwiązywaniu skomplikowanych, rzeczywistych problemów.
\end{document}