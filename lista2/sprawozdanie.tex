\documentclass[a4paper,14pt]{report}
\usepackage[utf8]{inputenc}
\usepackage[T1]{fontenc}
\usepackage{titlesec}
\usepackage{float}

\renewcommand{\contentsname}{Spis treści}

\titleformat{\chapter}[display]
  {\normalfont\bfseries}{}{0pt}{\Huge}

\title{Metody optymalizacji Lista 2}
\author{Radosław Wojtczak, numer indeksu: 254607}
\date{07.05.2023}
\begin{document}
\maketitle
\tableofcontents
\chapter{Zadanie 1}
\section{Wprowadzenie}
    Treść zadania: \\
    Tartak produkuje deski o standardowej szerokości 22 cali (każda deska ma ustaloną długość). Klienci firmy zamawiają jednak deski o mniejszej szerokości (i o tej samej długości,
jak deski o standardowej szerokości). Aktualne zamówienia opiewają na 110 desek o szerokości 7 cali, 120 desek o szerokości 5 cali i 80 desek o szerokości 3 cali. Deski o mniejszej
szerokości są odcinane z desek o standardowej szerokości. Na przykład firma może podjąć
decyzję o przecięciu dużej deski na dwie deski po 7 cali i jedną deskę o szerokości 5 cali.
W tym przypadku z deski standardowej tracona jest listwa o szerokości 3 cali. Firma chce
wykonać zamówienie w ten sposób, aby zminimalizować ilość odpadów
\section{Model}
    
\section{Wyniki} 

\chapter{Zadanie 2}
\section{Wprowadzenie}
    Celem zadania jest znalezienie odpowiedniego harmonogramu, według którego należy 
    wykonać zadania na jednej maszynie. Problem ów jest znany w literaturze pod nazwą 
    "single machine scheduling". Dokładniej formuując problem, mamy zadany zbiór
    $J = \{1,..,n\}$ składający się z $n$ zadań. Dla każdego zadania $j \in J $ definiowane są następujące własności:
    \begin{itemize}
        \item Czas potrzebny na wykonanie zadania ($p_{j}$)
        \item Priorytet zadania, określany też jako waga ($w_{j}$)
        \item Moment gotowości zadania ($r_{j}$)
    \end{itemize}
    Celem programu jest utworzenie harmonogramu minimalizującego iloczyn czasu zakończenia 
    danego zadania oraz jego priorytetu.
\section{Model}
    W ramach utworzenia modelu zdefiniowano następujące zmienne:
    \begin{itemize}
        \item (max\_T = maximum(r) + sum(p) + 1) - maksymalnie czas pracy maszyny
        \item jobs = $\{1,...,n\}$ - wektor przechowujący indeksy poszczególnych prac
        \item times = $\{1,...,max\_T\}$ - wektor przechowujący stany maszyny w danych momentach 
    \end{itemize}
    To na co należy zwrócić uwagę to fakt, że w ramach rozpatrywanego zadania traktujemy czas w sposób dyskretny.
    Zdefiniowawszy wszystkie zmienne pomocnicze poniżej przedstawiono definijcę zmiennych decyzyjnych.
    \subsection{Zmienne decyzyjne}
        \begin{equation}
            x_{jt} : j \in jobs, t \in times, x_{jt} \in \{0,1\}
        \end{equation}
        Zmienną decyzyjną jest macierz przechowująca momenty rozpoczęcia każdego z zadań.
        Rozpoczęcie zadania oznaczane jest przy pomocy liczby \textbf{1}, pozostałe wartości 
        są zerowane.
    \subsection{Ograniczenia}
        \begin{itemize}
            \item Każda praca może mieć tylko jeden moment rozpoczęcia 
                \begin{equation}
                    \sum_{j \in jobs} (\sum_{t \in times} x_{jt} = 1 )
                \end{equation}
            \item j-te zadanie może rozpocząć nie wcześniej, niż wskazuje na to wartość $r_{j}$
                \begin{equation}
                    \sum_{j \in jobs} (\sum_{t \in times} x_{jt}*t \geq r_{j})
                \end{equation}
            \item Maksymalnie jedno zadanie może być wykonywane w danej jednostce czasu
                \begin{equation}
                    \sum_{t \in times} (\sum_{j \in jobs, s \in max\{0,t-p_{j}+1 \}, t} x_{js} <= 1)
                \end{equation}
        \end{itemize}
    \subsection{Funkcja celu}
        Funkcją celu jest minimalizacja iloczynu czasu zakończenia 
        danego zadania oraz jego priorytetu.
        \begin{equation}
            \sum_{t \in times, j \in jobs} w_{j} * (t+p_{j}) * x_{jt}
        \end{equation}
\section{Wyniki}
    Zaimplementowany model został zaimplementowany dla następujących danych:
    \begin{itemize}
        \item $n = 6$
        \item $p = [2,3,4,1,3,2] $
        \item $w = [3,2,1,4,4,6] $
        \item $r = [2,1,1,0,0,4] $
    \end{itemize}
    Wyniki prezentują się w sposób następujący:
    \begin{table}[H]
        \centering
        \begin{tabular}{|c | c |} 
        \hline
        Zadanie & Czas rozpoczęcia\\
        \hline
        1 & 6 \\
        2 & 8 \\
        3 & 11 \\
        4 & 0 \\
        5 & 1 \\
        6 & 4 \\
        \hline
        \end{tabular}
        \caption{Numer zadania wraz z momentem rozpoczęcia dla wybranego egzemplarza problemu}
    \end{table}
    Ponadto funkcja celu została obliczona i jej wartość wynosi \textbf{117}.


\chapter{Zadanie 3}
\section{Wprowadzenie}
    Zadanie to jest naturalnym rozszerzeniem poprzedniego zadania. Tym razem mając zbiór zadań $J = \{1,..,n\}$ składający się z $n$ zadań, 
    tworząc harmonogram mamy do dyspozycji $m$ maszyn. Ponadto zrezygnowano z wag oraz momentu gotowości zadań na rzecz relacji pierwszeństwa.
    Jeśli dwa zadania $(i,j)$ są ze sobą w relacji to zadanie j nie może się rozpocząć przed ukończeniem zadania i. Zgodnie z treścią zadania, relacja 
    ta jest oznaczana przy pomocy symbolu → (i → j). Celem zadania jest
    znalezienie dopuszczalnego harmonogramu, który minimalizuje całkowity czas potrzebny do wykonania wszystkich zadań. 
    Czas ten oznaczono jako $c\_max$.
\section{Model}
    W celu wykonania zadania, podobnie jak w poprzednim zadaniu, wprowadzono dodatkowe zmiennej:
    \begin{itemize}
        \item (max\_T = sum(p) + 1) - maksymalnie czas pracy maszyny
        \item jobs = $\{1,...,n\}$ - wektor przechowujący indeksy poszczególnych prac
        \item times = $\{1,...,max\_T\}$ - wektor przechowujący stany maszyny w danych momentach 
    \end{itemize}
    \subsection{Zmienne decyzyjne}
    \begin{equation}
        x_{jt} : j \in jobs, t \in times, x_{jt} \in \{0,1\}
    \end{equation}
    Zmienną decyzyjną jest macierz przechowująca momenty rozpoczęcia każdego z zadań.
    Rozpoczęcie zadania oznaczane jest przy pomocy liczby \textbf{1}, pozostałe wartości 
    są zerowane.
    Ponadto zmienną decyzyjną jest zmienna $c\_max \geq 0$ informująca o czasie niezbędnym do wykoniania
    wszystkich zadań.
    \subsection{Ograniczenia}
        \begin{itemize}
            \item Wymagany czas nie może być mniejszy niż suma rozpoczęcia ostatniej pracy na maszynie oraz czasu jej trwania
            \begin{equation}
                \sum_{j \in jobs} (\sum_{t \in times} x_{jt} * (p_{j}+t)) <= c\_max
            \end{equation}
            \item Każda praca może mieć tylko jeden moment rozpoczęcia 
            \begin{equation}
                \sum_{j \in jobs} (\sum_{t \in times} x_{jt} = 1 )
            \end{equation}
            \item Zachowanie relacji pierwszeństwa między zadaniami
            \begin{equation}
                (\forall (j,relation) \in precedence) (\forall k \in relation) (\sum_{t \in times} (p_{j} + t)*x_{jt} \leq \sum_{t \in times} t*x_{kt}) 
            \end{equation}
            \item W jednej jednostce czasu nie może być więcej aktywny zadań niż dostepnych maszyn
            \begin{equation}
                \sum_{t \in times} (\sum_{j \in jobs, s \in max\{0,t-p_{j}+1 \}, t} x_{js} <= m)
            \end{equation}
        \end{itemize}
    \subsection{Funkcja celu}
    Funkcją celu jest minimalizacja zmiennej $c_max$.
    \begin{equation}
        min c_max
    \end{equation}
\section{Wyniki}
    Działanie programu zostało przetestowane dla następujących danych zaczerpniętych
    z treści zadania
    \begin{itemize}
        \item n = 9
        \item m = 3
        \item p = [1, 2, 1, 2, 1, 1, 3, 6, 2] 
        \item precedence = 1 → [4], 2 → [4,5], 3 → [4,5], 4 → [6,7], 5 → [7,8], 6 → [9], 7 → [9]
    \end{itemize}
\end{document}